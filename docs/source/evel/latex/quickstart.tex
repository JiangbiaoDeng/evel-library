\hypertarget{quickstart_qs_intro}{}\section{Introduction}\label{quickstart_qs_intro}
This Quick-\/\+Start section describes how to\+:


\begin{DoxyItemize}
\item Install and compile the supplied library code
\item Integrate an existing project to use the E\+V\+EL library
\end{DoxyItemize}\hypertarget{quickstart_qs_install}{}\section{Installation}\label{quickstart_qs_install}
The library is supplied as a source-\/code compressed-\/tar file. It is straightforward to install and build to integrate with an existing or new development project.\hypertarget{quickstart_qs_unpack}{}\subsection{Unpack the Source Code}\label{quickstart_qs_unpack}
The file should unpacked into your development environment\+: 
\begin{DoxyCode}
$ mkdir evel
$ cd evel
$ tar zxvf evel-library-package.tgz
\end{DoxyCode}
 \hypertarget{quickstart_qs_depend}{}\subsubsection{Satisfy Dependencies}\label{quickstart_qs_depend}
Note that all commands in this section are based on Cent\+OS package management tools and you may need to substitute the appropriate tools/packages for your distribution, for example {\ttfamily apt-\/get} for Ubuntu.

Ensure that G\+CC development tools are available.


\begin{DoxyCode}
$ sudo yum install gcc
\end{DoxyCode}
 Additionally, the library has a dependency on the c\+U\+RL library, so you\textquotesingle{}ll need the development tools for lib\+Curl installed. (At runtime, only the runtime library is required, of course.)


\begin{DoxyCode}
$ sudo yum install libcurl-devel
\end{DoxyCode}
 If you wish to make the project documentation, then Doxygen and Graphviz are required. (Again, this is only in the development environment, not the runtime environment!)


\begin{DoxyCode}
$ sudo yum install doxygen graphviz
\end{DoxyCode}


Note that some distributions have quite old versions of Doxygen by default and it may be necessary to install a later version to use all the features.

If you want to build P\+D\+Fs from the La\+TeX you will need a texlive install.


\begin{DoxyCode}
$ sudo yum install texlive
\end{DoxyCode}
\hypertarget{quickstart_qs_build}{}\subsubsection{Test Build}\label{quickstart_qs_build}
Make sure that the library makes cleanly\+:


\begin{DoxyCode}
$ cd bldjobs
$ make
Making dependency file evel\_unit.d for evel\_unit.c
Making dependency file evel\_test\_control.d for evel\_test\_control.c
Making dependency file evel\_demo.d for evel\_demo.c
Making dependency file jsmn.d for jsmn.c
Making dependency file evel\_logging.d for evel\_logging.c
Making dependency file evel\_event\_mgr.d for evel\_event\_mgr.c
Making dependency file evel\_internal\_event.d for evel\_internal\_event.c
Making dependency file evel\_throttle.d for evel\_throttle.c
Making dependency file evel\_syslog.d for evel\_syslog.c
Making dependency file evel\_strings.d for evel\_strings.c
Making dependency file evel\_state\_change.d for evel\_state\_change.c
Making dependency file evel\_scaling\_measurement.d for evel\_scaling\_measurement.c
Making dependency file evel\_signaling.d for evel\_signaling.c
Making dependency file evel\_service.d for evel\_service.c
Making dependency file evel\_reporting\_measurement.d for evel\_reporting\_measurement.c
Making dependency file evel\_json\_buffer.d for evel\_json\_buffer.c
Making dependency file evel\_other.d for evel\_other.c
Making dependency file evel\_option.d for evel\_option.c
Making dependency file evel\_mobile\_flow.d for evel\_mobile\_flow.c
Making dependency file evel\_fault.d for evel\_fault.c
Making dependency file evel\_event.d for evel\_event.c
Making dependency file double\_list.d for double\_list.c
Making dependency file ring\_buffer.d for ring\_buffer.c
Making dependency file metadata.d for metadata.c
Making dependency file evel.d for evel.c
Making evel.o from evel.c
Making metadata.o from metadata.c
Making ring\_buffer.o from ring\_buffer.c
Making double\_list.o from double\_list.c
Making evel\_event.o from evel\_event.c
Making evel\_fault.o from evel\_fault.c
Making evel\_mobile\_flow.o from evel\_mobile\_flow.c
Making evel\_option.o from evel\_option.c
Making evel\_other.o from evel\_other.c
Making evel\_json\_buffer.o from evel\_json\_buffer.c
Making evel\_reporting\_measurement.o from evel\_reporting\_measurement.c
Making evel\_service.o from evel\_service.c
Making evel\_signaling.o from evel\_signaling.c
Making evel\_scaling\_measurement.o from evel\_scaling\_measurement.c
Making evel\_state\_change.o from evel\_state\_change.c
Making evel\_strings.o from evel\_strings.c
Making evel\_syslog.o from evel\_syslog.c
Making evel\_throttle.o from evel\_throttle.c
Making evel\_internal\_event.o from evel\_internal\_event.c
Making evel\_event\_mgr.o from evel\_event\_mgr.c
Making evel\_logging.o from evel\_logging.c
Making jsmn.o from jsmn.c
Linking API Shared Library
Linking API Static Library
Making evel\_demo.o from evel\_demo.c
Making evel\_test\_control.o from evel\_test\_control.c
Linking EVEL demo
Making EVEL training
$
\end{DoxyCode}
 You should now be able to run the demo C\+LI application. Since it will want to dynamically link to the library that you\textquotesingle{}ve just made, you will need to set your {\ttfamily L\+D\+\_\+\+L\+I\+B\+R\+A\+R\+Y\+\_\+\+P\+A\+TH} appropriately first. Make sure that you specify your actual directory paths correctly in the following\+:


\begin{DoxyCode}
$ export LD\_LIBRARY\_PATH=$LD\_LIBRARY\_PATH:/home/centos/evel/libs/x86\_64
$ ../output/x86\_64/evel\_demo
evel\_demo [--help]
          --fqdn <domain>
          --port <port\_number>
          [--path <path>]
          [--topic <topic>]
          [--username <username>]
          [--password <password>]
          [--https]
          [--cycles <cycles>]
          [--nothrott]

Demonstrate use of the ECOMP Vendor Event Listener API.

  -h         Display this usage message.
  --help

  -f         The FQDN or IP address to the RESTful API.
  --fqdn

  -n         The port number the RESTful API.
  --port

  -p         The optional path prefix to the RESTful API.
  --path

  -t         The optional topic part of the RESTful API.
  --topic

  -u         The optional username for basic authentication of requests.
  --username

  -w         The optional password for basic authentication of requests.
  --password

  -s         Use HTTPS rather than HTTP for the transport.
  --https

  -c         Loop <cycles> times round the main loop.  Default = 1.
  --cycles

  -v         Generate much chattier logs.
  --verbose

  -x         Exclude throttling commands from demonstration.
  --nothrott

$
\end{DoxyCode}
 Assuming that all worked as expected, you are ready to start integrating with your application. It probably makes sense to make the L\+D\+\_\+\+L\+I\+B\+R\+A\+R\+Y\+\_\+\+P\+A\+TH change above permanent by incorporating it into your {\ttfamily .bash\+\_\+profile} file.\hypertarget{quickstart_qs_build_docs}{}\subsubsection{Project Documentation}\label{quickstart_qs_build_docs}
The source comes with its own documentation included. The documentation can be built using the {\ttfamily docs} target in the Makefile. By default this builds H\+T\+ML and La\+TeX documentation, the latter being used to prepare P\+D\+Fs.

To make the documentation\+: 
\begin{DoxyCode}
$ cd bldjobs
$ make docs
Cleaning docs...
Making Doxygen documentation
$ 
\end{DoxyCode}


There is a make target that is intended to install the documentation on a \char`\"{}team server\char`\"{} -\/ it will need adaptation for your team\textquotesingle{}s environment -\/ see the {\ttfamily docs\+\_\+install} target in the Makefile\+:


\begin{DoxyCode}
$ make docs\_install
Cleaning docs...
Making Doxygen documentation
Copying docs to team web-server...
Enter passphrase for key '/data/home/.ssh/id\_rsa':
annotated.html                           100% 8088     7.9KB/s   00:00    
arrowdown.png                            100%  246     0.2KB/s   00:00    
arrowright.png                           100%  229     0.2KB/s   00:00    
  ...
$
\end{DoxyCode}
\hypertarget{quickstart_qs_integrate}{}\section{Project Integration}\label{quickstart_qs_integrate}
There are two key steps to the integration which have to be undertaken\+:


\begin{DoxyItemize}
\item Initialization/\+Termination of the library.
\item Creation \& posting of individual events.
\end{DoxyItemize}

Additionally, it may be necessary to consider changes to the E\+V\+EL library\textquotesingle{}s source code if assumptions made by the library are either not satisfied or inconvenient. In particular\+:


\begin{DoxyItemize}
\item If the project already uses libcurl then the global initialization of the library should be removed from the {\itshape E\+V\+EL Library}.
\item The {\itshape E\+V\+EL Library} uses {\ttfamily syslog} for event logging. If the project uses a different event logging process, then E\+V\+EL\textquotesingle{}s event logging macros should be rewritten appropriately.
\end{DoxyItemize}

These steps are considered in the \hyperlink{quickstart_qs_normal_use}{Normal Use} and \hyperlink{quickstart_qs_adaptation}{E\+V\+EL Adaptation} sections below.\hypertarget{quickstart_qs_normal_use}{}\subsection{Normal Use}\label{quickstart_qs_normal_use}
The {\itshape E\+V\+EL Library} should be integrated with your project at a per-\/process level\+: each process is an independent client of the E\+C\+O\+MP Vendor Event Listener A\+PI.\hypertarget{quickstart_qs_initialize}{}\subsubsection{Initialization}\label{quickstart_qs_initialize}
The {\itshape E\+V\+EL Library} should be initialized before the process becomes multi-\/threaded. This constraint arises from the use of libcurl which imposes the constraint that initialization occurs before the system is multi-\/threaded. This is described in more detail in the libcurl documentation for the \href{https://curl.haxx.se/libcurl/c/curl_global_init.html}{\tt curl\+\_\+global\+\_\+init} function.

Initialization stores configuration of the Vendor Event Listener A\+PI\textquotesingle{}s details, such as the F\+Q\+DN or IP address of the service, so the initializing process must have either extracted this information from its configuration or have this information \char`\"{}hard-\/wired\char`\"{} into the application, so that it is available at the point the {\ttfamily \hyperlink{evel_8c_a059033f948f5bf406fedb4a7ef1966da}{evel\+\_\+initialize()}} function is called\+:


\begin{DoxyCode}
\textcolor{preprocessor}{#include "\hyperlink{evel_8h}{evel.h}"}
...
if (\hyperlink{evel_8c_a059033f948f5bf406fedb4a7ef1966da}{evel\_initialize}(api\_fqdn,
                    api\_port,
                    api\_path,
                    api\_topic,
                    api\_secure,
                    \textcolor{stringliteral}{"Alice"},
                    \textcolor{stringliteral}{"This isn't very secure!"},
                    \hyperlink{evel_8h_a304eae0d024005dc4c7031bdd774d64aa4fbe43c85c55ae14048a2dc7fc5bfcc8}{EVEL\_SOURCE\_VIRTUAL\_MACHINE},
                    \textcolor{stringliteral}{"EVEL demo client"},
                    verbose\_mode))
\{
  fprintf(stderr, \textcolor{stringliteral}{"Failed to initialize the EVEL library!!!"});
  exit(-1);
\}
...
\end{DoxyCode}
 Once initialization has occurred successfully, the application may raise events and may also use the logging functions such as \hyperlink{evel_8h_aeadebff1d4d408323d3992a2771510c7}{E\+V\+E\+L\+\_\+\+I\+N\+F\+O()}.

Initialization is entirely local (there is no interaction with the service) so it is very unlikely to fail, unless the application environment is seriously degraded.\hypertarget{quickstart_qs_generate}{}\subsubsection{Event Generation}\label{quickstart_qs_generate}
Generating events is a two stage process\+:


\begin{DoxyEnumerate}
\item Firstly, the {\itshape E\+V\+EL Library} is called to allocate an event of the correct type.
\begin{DoxyItemize}
\item If this is successful, the caller is given a pointer to the event.
\item All mandatory fields on the event are provided to this factory function and are thereafter immutable.
\item The application may add any necessary optional fields to the event, using the pointer previously returned.
\end{DoxyItemize}
\item The event is sent to the J\+S\+ON A\+PI using the \hyperlink{evel_8h_a209d2e5dbffe9e11ac79ae140f4a81bd}{evel\+\_\+post\+\_\+event()} function.
\begin{DoxyItemize}
\item At this point, the application relinquishes all responsibility for the event\+:
\begin{DoxyItemize}
\item It will be posted to the J\+S\+ON A\+PI, if possible.
\item Whether or not the posting is successful, the memory used will be freed.
\end{DoxyItemize}
\end{DoxyItemize}
\end{DoxyEnumerate}

In practice this looks like\+:


\begin{DoxyCode}
\textcolor{preprocessor}{#include "\hyperlink{evel_8h}{evel.h}"}
...

\textcolor{comment}{/***************************************************************************/}
\textcolor{comment}{/* Create a new Fault object, setting mandatory fields as we do so...      */}
\textcolor{comment}{/***************************************************************************/}
fault = \hyperlink{evel_8h_a6dd9b72800e61e661dc133b588aab892}{evel\_new\_fault}(\textcolor{stringliteral}{"My alarm condition"},
                       \textcolor{stringliteral}{"It broke very badly"},
                       \hyperlink{evel_8h_ad1bf6807fa6710332251611207490484af656e2fbfee66b25faa06447799a9011}{EVEL\_PRIORITY\_NORMAL},
                       \hyperlink{evel_8h_ad28dcc9cce27ecb7e4c9020107a6db9cac454e36d0a4678cd7ae2de725f513b61}{EVEL\_SEVERITY\_MAJOR});
\textcolor{keywordflow}{if} (fault != NULL)
\{
  \textcolor{comment}{/*************************************************************************/}
  \textcolor{comment}{/* We have a Fault object - add some optional fields to it...            */}
  \textcolor{comment}{/*************************************************************************/}
  \hyperlink{evel_8h_a08233f1e66ab6cdb0998fb3f5495ac45}{evel\_fault\_type\_set}(fault, \textcolor{stringliteral}{"Bad things happen..."});
  \hyperlink{evel_8h_a60ccecf94decb577e09fac9ee34358f2}{evel\_fault\_interface\_set}(fault, \textcolor{stringliteral}{"My Interface Card"});
  \hyperlink{evel_8h_a07076c0fa1d2941102f957af3b47df44}{evel\_fault\_addl\_info\_add}(fault, \textcolor{stringliteral}{"name1"}, \textcolor{stringliteral}{"value1"});
  \hyperlink{evel_8h_a07076c0fa1d2941102f957af3b47df44}{evel\_fault\_addl\_info\_add}(fault, \textcolor{stringliteral}{"name2"}, \textcolor{stringliteral}{"value2"});

  \textcolor{comment}{/*************************************************************************/}
  \textcolor{comment}{/* Finally, post the Fault.  In practice this will only ever fail if     */}
  \textcolor{comment}{/* local ring-buffer is full because of event overload.                  */}
  \textcolor{comment}{/*************************************************************************/}
  evel\_rc = \hyperlink{evel_8h_a209d2e5dbffe9e11ac79ae140f4a81bd}{evel\_post\_event}((\hyperlink{structevent__header}{EVENT\_HEADER} *)fault);
  \textcolor{keywordflow}{if} (evel\_rc != \hyperlink{evel_8h_ae1f68b0d21fb98defcec46e99208a03eaa9369500f041c2898e5c2b4f07b6d152}{EVEL\_SUCCESS})
  \{
    \hyperlink{evel_8h_afdddf91265803ed3fd502fedd7dabebf}{EVEL\_ERROR}(\textcolor{stringliteral}{"Post failed %d (%s)"}, evel\_rc, \hyperlink{evel_8h_acc6a1997021796b0a09fcdb8e89ea384}{evel\_error\_string}());
  \}
\}
...
\end{DoxyCode}
 \hypertarget{quickstart_qs_event_types}{}\subsubsection{Event Types}\label{quickstart_qs_event_types}
The {\itshape E\+V\+EL Library} supports the following types of events\+:


\begin{DoxyEnumerate}
\item Faults

These represent the {\bfseries fault} domain in the event schema.
\item Measurements

These represent the {\bfseries measurements\+For\+Vf\+Scaling} domain in the event schema.
\item Reports

This is an experimental type, designed to allow V\+N\+Fs to report application-\/level statistics unencumbered with platform measurements. The formal AT\&T schema has been updated to include this experimental type as {\bfseries measurements\+For\+Vf\+Reporting}.
\item Mobile Flow

These represent the {\bfseries mobile\+Flow} domain in the event schema.
\item Other

These represent the {\bfseries other} domain in the event schema.
\item Service Events

These represent the {\bfseries service\+Events} domain in the event schema.
\item Signaling

These represent the {\bfseries signaling} domain in the event schema.
\item State Change

These represent the {\bfseries state\+Change} domain in the event schema.
\item Syslog

These represent the {\bfseries syslog} domain in the event schema.
\end{DoxyEnumerate}\hypertarget{quickstart_qs_throttling}{}\subsubsection{Throttling}\label{quickstart_qs_throttling}
The {\itshape E\+V\+EL library} supports the following command types as defined in the J\+S\+ON A\+PI\+:


\begin{DoxyEnumerate}
\item command\+Type\+: throttling\+Specification

This is handled internally by the E\+V\+EL library, which stores the provided throttling specification internally and applies it to all subsequent events.
\item command\+Type\+: provide\+Throttling\+State

This is handled internally by the E\+V\+EL library, which returns the current throttling specification for each domain.
\item command\+Type\+: measurement\+Interval\+Change

This is handled by the E\+V\+EL library, which makes the latest measurement interval available via the \hyperlink{evel__throttle_8c_a5d0335c5b5e1220473b92497800ec036}{evel\+\_\+get\+\_\+measurement\+\_\+interval} function. The application is responsible for checking and adhering to the latest provided interval.
\end{DoxyEnumerate}\hypertarget{quickstart_qs_termination}{}\subsubsection{Termination}\label{quickstart_qs_termination}
Termination of the {\itshape E\+V\+EL Library} is swift and brutal! Events in the buffer at the time are \char`\"{}dropped on the floor\char`\"{} rather than waiting for the buffer to deplete first.


\begin{DoxyCode}
\textcolor{preprocessor}{#include "\hyperlink{evel_8h}{evel.h}"}
...

\textcolor{comment}{/***************************************************************************/}
\textcolor{comment}{/* Shutdown the library.                                                   */}
\textcolor{comment}{/***************************************************************************/}
evel\_terminate();

...
\end{DoxyCode}
\hypertarget{quickstart_qs_adaptation}{}\subsection{E\+V\+E\+L Adaptation}\label{quickstart_qs_adaptation}
The {\itshape E\+V\+EL Library} is relatively simple and should be easy to adapt into other project environments.

\subsubsection*{Libc\+U\+RL Lifecycle}

There are two circumstances where initialization of libcurl may be required\+:


\begin{DoxyEnumerate}
\item If libcurl is used by the project already, and therefore already takes responsibility of its initialization, then the libcurl initialization and termination functions should be removed from \hyperlink{evel_8c_a059033f948f5bf406fedb4a7ef1966da}{evel\+\_\+initialize()} and \hyperlink{evel_8c_ab3f6225ddb9c5113d74503d4fcd17e5b}{evel\+\_\+terminate()} respectively.
\item If the project is unable to satisfy the constraint that libcurl initialization takes place in a single-\/threaded environment at the point that the {\itshape E\+V\+EL Library} can be initialized (for example, if MT code is necessary to read the configuration parameters required for {\itshape E\+V\+EL Library} initialization) then it may be necessary to extract the libcurl functions and call them separately, earlier in the program\textquotesingle{}s operation.
\end{DoxyEnumerate}

\subsubsection*{Event Logging}

The {\itshape E\+V\+EL Library} uses {\ttfamily syslog} for logging. If this is inappropriate then the \hyperlink{evel_8h_ab6073d0ad433cc7e272b66952063a4fb}{log\+\_\+debug()} and \hyperlink{evel_8h_a7e62b4de31e7c058e1534d0ac6040e98}{log\+\_\+initialize()} functions should be rewritten accordingly.

{\bfseries Note}\+: it would be a really bad idea to use the {\itshape E\+V\+EL Library} itself for this logging function. \href{https://en.wikipedia.org/wiki/Turtles_all_the_way_down}{\tt Turtles all the way down...} 